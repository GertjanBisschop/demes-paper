% LaTeX rebuttal letter example.
%
% Copyright 2019 Friedemann Zenke, fzenke.net
%
% Based on examples by Dirk Eddelbuettel, Fran and others from
% https://tex.stackexchange.com/questions/2317/latex-style-or-macro-for-detailed-response-to-referee-report
%
% Licensed under cc by-sa 3.0 with attribution required.

\documentclass[11pt]{article}
\usepackage[utf8]{inputenc}
\usepackage{lipsum} % to generate some filler text
\usepackage{fullpage}

% import Eq and Section references from the main manuscript where needed
% \usepackage{xr}
% \externaldocument{manuscript}

% package needed for optional arguments
\usepackage{xifthen}
% define counters for reviewers and their points
\newcounter{reviewer}
\setcounter{reviewer}{0}
\newcounter{point}[reviewer]
\setcounter{point}{0}

% This refines the format of how the reviewer/point reference will appear.
\renewcommand{\thepoint}{\thereviewer.\arabic{point}}

% command declarations for reviewer points and our responses
\newcommand{\reviewersection}{\stepcounter{reviewer} \bigskip \hrule
                  \section*{Reviewer \thereviewer}}

\newenvironment{point}
   {\refstepcounter{point} \bigskip \noindent {\textbf{Reviewer~Point~\thepoint} } ---\ }
   {\par }

\newcommand{\shortpoint}[1]{\refstepcounter{point}  \bigskip \noindent
    {\textbf{Reviewer~Point~\thepoint} } ---~#1\par }

\newenvironment{reply}
   {\medskip \noindent \begin{sf}\textbf{Reply}:\  }
   {\medskip \end{sf}}

\newcommand{\shortreply}[2][]{\medskip \noindent \begin{sf}\textbf{Reply}:\  #2
    \ifthenelse{\equal{#1}{}}{}{ \hfill \footnotesize (#1)}%
    \medskip \end{sf}}

\begin{document}

\section*{Response to the editor}
% General intro text goes here
Thank you for considering this manuscript for publication.

\subsection*{Specific comments}
% We address your specific comments below:
% % \reviewersection

\textit{
Demes is an important advance to standardizing the implementation of popgen
models, and so a very welcome contribution. The reviewers give a number of very
useful suggestions. I leave it up to you how you proceed with incorporating
changes based on these comments, as you will have more of a sense of how the
comments overlap those from other users, but please do write a response to all
of the comments.
}


\section*{Response to the reviewers}
% General intro text goes here
We thank the reviewers for their close reading of our manuscript and
insightful comments. In the following we address the points raised
in turn.

\reviewersection

\begin{point}
This manuscript presents the Demes data model and related libraries ecosystem.
I have been using it for some time and was happy to review the paper as I find
it really useful, intuitive and well designed (as are a lot of tools developed
by those authors and the whole community of the popsim consortium and tskit).
The main usage I have for it at the moment was to switch effortlessly between
msprime and SLiM using the same demographic model, loved that. But as described
in the paper, it has a lot of other qualities. Overall, this format is a really
useful addition to the field and which I hope will be largely adopted.

I don't have much to add to this manuscript other than a few comments that I
hope will help answer a few questions that other readers could have. My main
comment below is that an example of the YAML format should be included in the
main text.
\end{point}
\begin{reply}
TODO
\end{reply}

\begin{point}
I usually like when there is a link to the software directly in the abstract
 in those kind of papers, it avoids scrolling through it as links are
sometimes in different places in those type of manuscripts.
\end{point}
\begin{reply}
TODO
\end{reply}

\begin{point}
I suggest including the example of the YAML format in the main text.
This is too sad that the center piece of this paper has been relegated to the appendix :(
Show us how easy it is to understand the model from the YAML format!
Maybe FigA2 is a bit too long, but FigA1 A and B could work well.
\end{point}
\begin{reply}
TODO
\end{reply}

\begin{point}
A fair question that could be asked by any potential user could be: Are there
  versions to this data format? There is no version argument in the YAML file,
what if the format changes in the future? How would my model format would be
recognized by the parser? I get that it is designed with stability in mind and
it is done well, but we've seen formats change for other things through time
(e.g. in bioinformatics) to adapt to new usage.
\end{point}
\begin{reply}
TODO
\end{reply}

\begin{point}
There might be the need to specify how the initialization time of the whole
  model is dealt with. This is kind of straightforward when thinking about
coalescent models for people used to it. But not for forward-time simulations.
Especially that in your examples there are no start\_time for the first
population. This might be hard for some users to wrap their head around.
\end{point}
\begin{reply}
TODO
\end{reply}

\begin{point}
p2 you discuss the defaults of CLI and input parameter formats but not of
  APIs. It could be said that APIs require some learning (of both the interface
and the given programming language for users not used to it). Learning a new
API each time you change simulator is also time consuming.
\end{point}
\begin{reply}
TODO
\end{reply}

\begin{point}
p2 "YAML format" you could specify that it is a data serialization language
\end{point}
\begin{reply}
TODO
\end{reply}

\begin{point}
p12 "time units is in generations" you could precise it is the default value here, if it is?
\end{point}
\begin{reply}
TODO
\end{reply}

\begin{point}
p12 "such as years, accompanied by the generation time". Might be worth
  detailing that it is needed as simulators work with generations and not
years, so there needs to be some kind of conversion.
\end{point}
\begin{reply}
TODO
\end{reply}

\begin{point}
p12 "function defined by start and end sizes" You should detail here what are
  the models of population expansion available and what is the default. Linear,
exponential, etc.?
\end{point}
\begin{reply}
TODO
\end{reply}

\begin{point}
p12 "S are born via [...]" missing that S is a proportion
\end{point}
\begin{reply}
TODO
\end{reply}

\begin{point}
p13-14 "Other implicit values [...]" until end of paragraph, this part with
  the example could be worth integrating into the main text as this is a major
thing to understand about the data model in my opinion.
\end{point}
\begin{reply}
TODO
\end{reply}

\begin{point}
Section A3: you could also detail that having common parsers allows for
  consistent and informative error messages when it comes to missing values or
issues in formatting.
\end{point}
\begin{reply}
TODO
\end{reply}

\begin{point}
p15 what's the difference between "genome annotations" and "sequence annotations"?
\end{point}
\begin{reply}
TODO
\end{reply}

\textit{Random thoughts I had while reading the paper, to discuss (or not)}

\begin{point}
I wondered if such format couldn't be used as a way to increase (spoken)
  language inclusivity. I guess it would be fairly easy to have dictionaries
including multiple translations for each keyword of the format in the parser.
That way you could write your model in Japanese or Spanish and it could still
be read without issue.
\end{point}
\begin{reply}
TODO
\end{reply}

\begin{point}
Do you think such plain text format also provide a better substrate for
versionning demographic models? Could be mentioned.
\end{point}
\begin{reply}
TODO
\end{reply}

\reviewersection

\begin{point}

\end{point}
\begin{reply}
TODO
\end{reply}

\begin{point}

\end{point}
\begin{reply}
TODO
\end{reply}

\end{document}
