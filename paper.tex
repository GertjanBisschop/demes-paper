\documentclass[11pt]{article}
\usepackage[round,comma,authoryear]{natbib}
\usepackage{fullpage}
\usepackage{authblk}
\usepackage{graphicx}

\usepackage[newfloat]{minted}
\usepackage{tcolorbox}
\usepackage{etoolbox}
\BeforeBeginEnvironment{minted}{\begin{tcolorbox}}
\AfterEndEnvironment{minted}{\end{tcolorbox}}
\usepackage{url}
\usepackage{fancyvrb}
\usepackage{caption}
\newenvironment{code}{\captionsetup{type=listing}\centering}{}
\SetupFloatingEnvironment{listing}{name=Demographic Model}
% can either input code directly in a minted environment
% or input from a source file using \inputminted

% local definitions
\newcommand{\msprime}[0]{\texttt{msprime}}
\newcommand{\stdpopsim}[0]{\texttt{stdpopsim}}
\newcommand{\demes}[0]{\texttt{demes}}
\newcommand{\Demes}[0]{\texttt{Demes}}
\newcommand{\YAML}[0]{\texttt{YAML}}
\newcommand{\moments}[0]{\texttt{moments}}
\newcommand{\dadi}[0]{\texttt{$\partial$a$\partial$i}}
\newcommand{\fwdpy}[0]{\texttt{fwdpy11}}
\newcommand{\slim}[0]{\texttt{SLiM}}
\newcommand{\gadma}[0]{\texttt{GADMA}}

\newcommand{\aprcomment}[1]{{\textcolor{blue}{APR: #1}}}
\newcommand{\jkcomment}[1]{{\textcolor{red}{JK: #1}}}

\begin{document}

\title{\demes: a pain-free specification for complex demography}
\author[*]{Graham Gower}
\author[*]{Aaron P. Ragsdale}
\author[ ]{...Others here...}
\author[**]{Jerome Kelleher}
\author[**]{Kevin Thornton}
\affil[*,**]{Denote equal contribution}
\maketitle

\abstract{
}

\section*{Introduction}

The ever-increasing amount of genetic sequencing data from genetically and
geographically diverse species and populations has allowed us to infer complex
demography and study life history at fine scales. An integral component to such
population genetics studies is simulation. Software to either simulate whole
genome sequences
\citep{thornton2014c++,kelleher2016efficient,haller2019slim,adrion2020community}
or informative summary statistics of diversity
\citep{gutenkunst2009inferring,kamm2017efficient,jouganous2017inferring} have
enabled the increasing complexity of genomic studies, with many software
packages able to handle large sample sizes, many interacting populations, and
deviations from random-mating models of panmixia.

The ability to simulate complex demographic scenarios, however, does not come
without its own set of obstacles and frustrations. First, specifying models
with many populations and parameters can be cumbersome and error-prone (e.g.,
\citep{ragsdale2020lessons}). It is time-consuming and tedious to implement and
then verify the correctness of such demographic models. Making matters worse,
each software package typically has its own application programming interface
(API), which makes it difficult to translate models between simulation
software.

Second, many software API are not particularly easy to read, especially when
there are a large number of demographic events and parameters. This not only
makes debugging difficult, but it can be difficult to even notice if a mistake
has been made. A common interface for unambiguously specifying demographic
models would reduce implementation errors and promote ease-of-use for many
simulation software packages.

Finally, simulated or inferred demographic models are regularly shared in the
literature, and for someone else to be able to reproduce or reimplement some
demographic scenario, it needs to be described fully and without ambiguity.
However, it is common for reported demographic models to lack clear or even
complete descriptions, hindering reproducibility. A human-readable description
for such models would facilitate their sharing and reimplementation.

Here, we present \demes, a specification of complex demographic models that is
both human readable and can be directly passed to a growing number of
simulation software. \Demes\ supports model specification in a high-level
\YAML\ format \citep{ben2009yaml} that is designed to be as easy as possible to
implement, read, and share. Models are then represented in an unambiguous
low-level format within Python that checks model validity. The initial Python
release of \demes\ (version 0.1, described here) supports \emph{static}
demographic descriptions, that is, models with fixed parameters. Future
releases will support [[drawing parameters from distributions, for example for
ABC-based inference, or describing ``dynamic`` models where parameters may be
fit to data, e.g. \moments\ or \dadi]].

\section*{The \demes\ specification}

\begin{itemize}
    \item Demographic models are comprised of one or more interacting
        populations, along with description, doi, time units and generation
        times.
    \item Populations (or demes) are treated as discrete collection of
        exchangeable individuals following a well-defined set of rules
    \item These rules include population sizes, times of existence, and other
        information about the mating system (e.g. selfing or cloning rates)
    \item Relationships between demes are defined by ancestors/descendents
        and possible continuous or instantaneous migration between coexisting
        demes
\end{itemize}

\subsection*{High-level model specification in \YAML}

\begin{itemize}
    \item YAML allows us to write this information in a human-readable and
        structured manner (lists, keys/items, etc)
    \item Implicit values, so that models can be written compactly
    \item Example of an IM Model~\ref{code:im_model}
\end{itemize}

\begin{code}
\begin{tcolorbox}
\inputminted[linenos,numbersep=5pt]{yaml}{models/IM.yaml}
\end{tcolorbox}
\captionof{listing}{An IM model.}
\label{code:im_model}
\end{code}

\subsection*{Unambiguous low-level representation}

\begin{itemize}
    \item The Python API, which can read in a YAML file or be built using the
        Builder
    \item Demographic model validitation
    \item Unambiguous, redundant information
    \item Designed to be programatically read by simulation software backends
\end{itemize}


\begin{table}
\begin{center}
\begin{tabular}{llr}
\msprime & \citep{kelleher2016efficient} & Coalescent simulator \\
\stdpopsim & \citep{adrion2020community} & Library of standardised simulations \\
\fwdpy & \citep{thornton2014c++} & Forward-time simulator \\
\moments & \citep{jouganous2017inferring,ragsdale2019models}
    & Demographic inference \\
\dadi & \citep{gutenkunst2009inferring}  & Demographic inference \\
\gadma & \citep{noskova2020gadma}  & Demographic inference \\
\end{tabular}
\end{center}
\caption{Software supporting Demes as an input or output format.
\jkcomment{This is a very rough first pass. It's not clear that a table is the
right format for presenting this information, maybe a description
list would be better instead?}}
\end{table}

\section*{Extensions and future directions}

\subsection*{Software support for \demes}
\aprcomment{at time of publication. TBD...}
\begin{itemize}
    \item \msprime: coalescent simulation \citep{kelleher2016efficient}
    \item \stdpopsim: a collection of maintained species and demographic models,
        which includes a large number of models specified using \demes
        \citep{adrion2020community}
    \item \fwdpy: forward-time Wright-Fisher simulation \citep{thornton2014c++}
    \item \moments: compute expected diversity statistics (SFS, LD) and perform
        demographic inference using summary statistics
        \citep{jouganous2017inferring,ragsdale2019models}
    \item \dadi: compute the expected SFS and perform demographic inference using
        summary statistics \citep{gutenkunst2009inferring}
\end{itemize}

\subsection*{Other stuff...}

\begin{itemize}
    \item \texttt{demesdraw} (\url{https://github.com/grahamgower/demesdraw.git})
    \item General framework for inference (have been working on that over in \moments,
        and perhaps it could be generalized to work with any simulation engine
    \item Defining distributions of parameter values, perhaps?
\end{itemize}

\bibliographystyle{plainnat}
\bibliography{paper}

\section*{Appendix}

\aprcomment{Perhaps and example of a many-population model, and API to simulate using
\msprime, \fwdpy, and \moments?}


\end{document}
