\documentclass[11pt]{article}
\usepackage[round,comma,authoryear]{natbib}
\usepackage{fullpage}
\usepackage{authblk}
\usepackage{graphicx}
\usepackage{booktabs}

\usepackage[newfloat]{minted}
\usepackage{tcolorbox}
\usepackage{etoolbox}
\BeforeBeginEnvironment{minted}{\begin{tcolorbox}}
\AfterEndEnvironment{minted}{\end{tcolorbox}}
\usepackage{url}
\usepackage{fancyvrb}
\usepackage{caption}
\newenvironment{code}{\captionsetup{type=listing}\centering}{}
\SetupFloatingEnvironment{listing}{name=Demographic Model}
% can either input code directly in a minted environment
% or input from a source file using \inputminted

% local definitions
\newcommand{\ms}[0]{\texttt{ms}}
\newcommand{\msprime}[0]{\texttt{msprime}}
\newcommand{\stdpopsim}[0]{\texttt{stdpopsim}}
\newcommand{\demes}[0]{\texttt{demes}}
\newcommand{\Demes}[0]{\texttt{Demes}}
\newcommand{\moments}[0]{\texttt{moments}}
\newcommand{\dadi}[0]{\texttt{$\partial$a$\partial$i}}
\newcommand{\fwdpy}[0]{\texttt{fwdpy11}}
\newcommand{\slim}[0]{\texttt{SLiM}}
\newcommand{\gadma}[0]{\texttt{GADMA}}

\newcommand{\aprcomment}[1]{{\textcolor{blue}{APR: #1}}}
\newcommand{\jkcomment}[1]{{\textcolor{red}{JK: #1}}}

\begin{document}

\title{Demes: a specification for interchanging complex demography}
% First authors
\author[1,$\star$]{Graham Gower}
\author[2,$\star$]{Aaron P. Ragsdale}

% Authors: please add your name and affiliation here, in alphabetical order.
% Second Authors
\author[3]{Ryan N. Gutenkunst}
\author[4]{Matthew Hartfield}
\author[5]{Ekaterina Noskova}
\author[3]{Travis J. Struck}
%Senior Authors
\author[6,$\dagger$,$\aleph$]{Jerome Kelleher}
\author[7,$\dagger$]{Kevin R. Thornton}

\affil[1]{Lundbeck GeoGenetics Centre, Globe Institute, University of Copenhagen}
\affil[2]{FIXME: Aaron's affiliation}
\affil[3]{Department of Molecular and Cellular Biology, University of Arizona}
\affil[4]{Institute of Evolutionary Biology, The University of Edinburgh}
\affil[5]{Computer Technologies Laboratory, ITMO University}
\affil[6]{Big Data Institute, Li Ka Shing Centre for Health Information and
    Discovery, University of Oxford}
\affil[7]{Ecology and Evolutionary Biology, University of California at Irvine}

\affil[$\star$]{Denotes shared first authorship, listed alphabetically}
\affil[$\dagger$]{Denotes shared senior authorship, listed alphabetically}
\affil[$\aleph$]{Denotes corresponding author}


\maketitle

\abstract{
Understanding the demographic history of populations is a
key goal in population genetics, and with improving methods
and data, ever more complex models are being estimated.
Demographic models of current interest (consisting of a set of discrete populations,
their sizes and growth rates, and continuous and pulse migrations
between those populations over a number of epochs) can require
dozens of parameters to fully describe. There is currently
no standard format to define such models, significantly
hampering  progress in the field. In particular, the important
task of translating the model descriptions in published
work into input suitable for population genetic simulators is labour intensive
and error prone.
We propose the Demes data model and file format,
built on widely used technologies,
to alleviate these issues. Demes provides a well-defined and unambigous
graphical model of populations and their properties that is straightforward to
implement in software, and a text file format that is designed for
simplicity and clarity.
We provide thoroughly tested implementations of Demes parsers in Python
and C, and showcase initial support in a number of simulators and inference
methods.
}

\section*{Introduction}

The ever-increasing amount of genetic sequencing data from genetically and
geographically diverse species and populations has allowed us to infer complex
demography and study life history at fine scales. An integral component to such
population genetics studies is simulation. Software to either simulate whole
genome sequences
\citep{thornton2014cpp,staab2015scrm,kelleher2016efficient,haller2019slim}
or informative summary statistics of diversity
\citep{gutenkunst2009inferring,kamm2017efficient,jouganous2017inferring} have
enabled the increasing complexity of genomic studies, with several software
packages capable of working with large sample sizes, many interacting populations, and
deviations from random-mating models of panmixia.
This ability to infer and simulate such complex demographic scenarios, however,
has highlighted a major shortcoming in community standards. The fragmented
landscape of different approaches to describing demographic models makes
it difficult to compare inferences made by different methods,
and to reliably simulate from previously inferred models.

Inference methods for demography usually output a set of
text files which describe the inferred model,
and nearly all methods use different file formats.
% Thus, any comparison
% between inferences requires these file formats to be parsed, with the user
% being left with the task of unifying the data models.
Inference results are typically reported in publications
via a combination of visual depiction,
a list of key parameters in tabular form and a discussion within the text.
Unfortunately there is often ambiguity in these descriptions and
implementing the precise model inferred for later simulation
is at best tedious and error prone~\citep{ragsdale2020lessons}
and occasionally impossible because of missing information.
The \stdpopsim\ project is a community-maintained catalog of species
information and demographic models designed to provide standardised
population genetic simulations~\citep{adrion2020community}.
The development process for adding demographic models to
\stdpopsim\ tries to avoid errors in model implementation by requiring two
independent implementations of each model
to be equivalent. While this process is
robust and has uncovered both significant
errors~\citep{ragsdale2020lessons} and ambiguities in
published models, it is labour intensive.
It would be far better if published results
were reported in an unambiguous standardised format that could be provided
directly as input to simulators.

Simulation is a core tool in population genetics, and a
large number of different methods have been developed
over the past three decades~\citep{carvajal2008simulation,liu2008survey,
arenas2012simulation,yuan2012overview,hoban2012computer}.
Simulations are based on highly idealised population models
and one of the key uses of inferred demographic histories
is to make simulations more realistic.
Simulation methods take three broad approches to specifying
the demographic model to simulate,
using either a command line
interface~\citep[e.g.][]{hudson2002generating,kern2016discoal},
a custom input file
format~\citep[e.g.][]{guillaume2006nemo,excoffier2011fastsimcoal,shlyakhter2014cosi2},
or an Application Programming Interface (API) to allow
models be defined programmatically~\citep[e.g.][]{
thornton2014cpp,kelleher2016efficient,becheler2019quetzal,haller2019slim}.
Command line interfaces are a concise way of expressing
demographic models, and the syntax defined by \ms~\citep{hudson2002generating}
is used by a number of different
simulators~\citep[e.g.][]{ewing2010msms,chen2009fast,staab2015scrm},
and the POPDemog visualisation method~\citep{zhou2018popdemog}.
However, this conciseness means that models of even intermediate complexity
are difficult for humans to understand, making errors highly likely.
Input parameter file formats for simulators allow the model specification
to be less terse and (potentially) also allow for documentation in the
form of comments. However, there is no standard format shared across
simulators, and the details of the demographic models are not separated
from other simulation parameters. The APIs use by individual simulators
to define demographic models are also not appropriate as an interchange
standard because of the dependency on both the programming language involved,
and the specific details of the simulator in question.

Here we present ``Demes'', a data model and file format
specification for complex demographic models developed by the PopSim
Consortium. The Demes data model precisely defines key parameters
of populations and their relationships over time in an extensible way,
and provides a way to explicitly encode all information relevant to
demography while avoiding repetition. This data model is implemented
in the widely used YAML format~\citep{ben2009yaml}, which provides a
good balance between human and machine readability.
The specification precisely
defines the required behaviour of implementations, ensuring that there
is no ambiguity of interpretation, and includes both a reference
implementation and an extensive suite of test examples and their expected
output. The initial software ecosystem includes high-quality
Python and C parser implementations, as well as utilities for verification
and visualisation of Demes models, and has been implemented in
several popular inference and simulation methods
(Table~\ref{tab-software}).
We hope that this
data model and file format will be widely adopted by the community,
such that users can expect to simulate directly from
inferred models with little or no programming effort.

\section*{Demes}

The Demes specification is a formal data model for describing
the properties of populations over time,
along with some metadata and provenance information.
The data model is based on the ubiquitous JSON~\citep{bray2017javascript}
standard, and the structure formally defined using
JSON Schema~\citep{wright2020json}.
The Demes JSON schema rigorously defines the data model,
specifying the hierarchical structure and the properties that are allowed in different
contexts along with their types and permissable ranges. This schema, and its
accompanying documentation, fully describe the entities in the model and their
relationships, and the required behaviour of implementations. Since the
schema is definitive, we will not recapitulate the details
here, but instead focus on the high level properties of the model and
the rationale behind key design decisions.

In Demes, demographic models consist of one or more interacting
% JK: not sure at all about this, but let's see how it goes for now.
populations (or ``demes"; to avoid confusion with the name of the
specification itself we will use the term ``population'' in this
discussion, in the understanding that the terms are interchangeable
for our purposes). Populations are defined as a collection of
exchangable individuals that exist for some period of time,
following a well-defined set of rules regarding population
sizes, mating systems, etc. Ancestor/descendent relationships
between populations can be defined, as well as
possible continuous or instantaneous migration between coexisting
populations. The history of each population is defined as a
series of ``epochs'' during which the population's parameters are
fixed.

The design of Demes is a balance between two partially competing requirements:
that (a) models should be easy for humans to understand and manipulate; and (b)
software processing Demes models should be provided with an unambiguous
representation that is straightforward to process. For efficiency of
understanding and avoidance of model specification error, we require a data
representation without redundancy (i.e., repetition of values). But, for the
simplicity of software working with the Demes model (and the avoidance of
programming error, or divergence in interpretations of the specification)
it is preferable to have an explicit representation, in
which all relevant values are readily available. Thus, we consider
three entities: the high-level, human readable, model specification;
the low-level, fully-qualified model specification; and the parser,
which is responsible for transforming the former into the latter.

\begin{figure}
\begin{minipage}{0.48\textwidth}
\begin{tcolorbox}
\inputminted[fontsize=\scriptsize,linenos,numbersep=5pt]{yaml}{models/IM.yaml}
\end{tcolorbox}
\begin{tcolorbox}
\includegraphics[width=\textwidth]{fig/IM}
\end{tcolorbox}
\end{minipage}\hfill
\begin{minipage}{0.48\textwidth}
\begin{tcolorbox}
\inputminted[fontsize=\scriptsize,linenos,numbersep=5pt]{yaml}{models/IM-resolved.yaml}
\end{tcolorbox}
\end{minipage}\\
\caption{\label{fig-example}
Example isolation-with-migration Demes model. (A) The compact non-redundant
representation expressed using YAML. (B) The same model in the fully-qualified
form. (C) A visual representation of the model using \texttt{demesdraw}.
The YAML descriptions in (A) and (B) correspond exactly to a JSON description,
but are much more concise.
}
\end{figure}


\subsection*{High-level model specification}

The high-level description of Demes models is focused on the efficiency of
human understanding and the avoidance of errors. We have adopted the widely
used YAML format~\citep{ben2009yaml} as the recommended means of interchanging
Demes models (see Figure~\ref{fig-example} for exampels). YAML is a data
serialisation language with an emphasis on simplicity, and interoperates well
with JSON (indeed, YAML 1.2 is a superset of JSON). We chose YAML over JSON
because although JSON is an excellent format for data interchange, it is
ill-suited for human understanding or manipulation. We also considered other
data formats such as TOML, but chose YAML because of its greater popularity and
software support. Since the Demes data model is defined in JSON Schema,
however, there is no formal dependency on YAML and implementations may choose
to use JSON directly if they wish (e.g., for greater efficiency).

Structurally, the high-level Demes specification encourages human understanding
by avoiding redundancy in the description where possible, and
by providing a mechanism for specifying default values that
are inherited hierarchically. Figure~\ref{fig-example} shows an example
Demes model expressed in both the high and low-level forms (both in
terms of YAML syntax).

\jkcomment{Finish up this section with a discussion of how we (a)
use default values defined in JSON schema; (b) define more
subtle implicit values in the specification itself; and (c) use the
defaults sections, all explified by Fig 1.}

\subsection*{Parsers}
The implicit value inference and default value propagation required to
fully resolve the high-level Demes model description is straightforward
to describe, but still requires significant effort to implement in
software. Moreover, there are many constraints on the data model
(for example, epoch \texttt{end\_time} values should be strictly increasing
within a deme), which must be enforced.
It would not be reasonable to require every program that
takes Demes models as input to implement this logic, as the programming
effort would be significant (limiting adoption)
and it is likely that if there were many implementations some would differ
in their details (harming the software ecosystem). We therefore define
a standard software entity as part of the specification (the parser),
which performs this task, and which can be shared by programs that
support demes as an input. By having relatively few (ideally, one
per programming language), high-quality Demes parsers available as
libraries the probability of divergences from the standard is
greatly reduced.

The Demes specification precisely defines the required behaviour of
parsers, and we also provide a reference implementation written
in Python to resolve any potential ambiguities and to provide a
helpful template for other implementations. We also provide an
extensive suite suite of examples and the expected outputs. In addition,
we have high-quality parser implementations in the Python and C
languages (published under liberal open source licences),
providing a solid foundation for the software ecosystem.


\subsection*{Unambiguous low-level representation}

\jkcomment{Not sure if there's much point in this as a separate section any
more? This could really be said in the previous section.}

Parsers output the low-level description of Demes models,
which is an explicit representation
of the model where all parameters have been assigned values, and
all data constraints have been checked. The output is formally defined
as JSON, but in practise parsers output an object model that
is suitable for the particular programming language, and
that can be used directly by the implementing program.


\subsection*{Design rationale}

This is an unstructured list of things we should mention in terms
of why certain decisions were made

\begin{itemize}
    \item Recent developments in software engineering have favoured
    static, declarative configuration formats. It feels like a good idea to
    have a Turing complete configuration format, but it turns out not to be.

    \item Drawing parameters from distributions. Why we are static.
    % https://github.com/popsim-consortium/demes-paper/issues/3

    \item Why we've avoided turning this into a model of everything, allowing
    for the specification of genome properties like recombination rates,
    etc.

    \item Why we've versioned the specification, and how we've planned
    for future extensions.

\end{itemize}

% Add a bit more space between rows.
\renewcommand{\arraystretch}{1.5}
\begin{table}
\begin{center}
\begin{tabular}{lp{12cm}}
\toprule
\multicolumn{2}{l}{Software infrastructure}\\
\midrule
\texttt{demes-python} &
    A Python library for loading, saving, and working with
    Demes models. Includes conversion utilities for existing
    demographic model descriptions such as
    \texttt{ms}~\citep{hudson2002generating}.\\

\texttt{demes-c} &
    A C library for parsing Demes YAML descriptions.
    (\url{https://github.com/grahamgower/demes-c.git}). \\

\texttt{demesdraw} &
    A Python library for visualising Demes models
    (\url{https://github.com/grahamgower/demesdraw.git}). \\

\midrule
\multicolumn{2}{l}{Methods using Demes as input/output format}\\
\midrule

% TOOL AUTHORS: please add a row to this table to summarise what the software
% does, and how it implements demes.

\dadi &
    Summary of what dadi does and how it implements Demes~\citep{gutenkunst2009inferring}.
    \\

\fwdpy &
    Forward-time Wright-Fisher simulation \citep{thornton2014cpp}\\

\gadma &
    Demographic inference \citep{noskova2020gadma}.\\

\moments &
    Compute expected diversity statistics (SFS, LD) and perform
    demographic inference using summary statistics
    \citep{jouganous2017inferring,ragsdale2019models}.\\

\msprime &
    Coalescent simulation of Demes models, including command line
    interface~\citep{kelleher2016efficient,kelleher2020coalescent}.\\

\stdpopsim &
    A community-maintained collection of empirical data for species
    and demographic models specified in Demes
    format~\citep{adrion2020community}.\\

\bottomrule
\end{tabular}
\end{center}
\caption{\label{tab-software}
Software support for Demes. We have included software infrastructure developed
for working with Demes models (parsing, validation, visualisation, etc)
as well as all known downstream software that implements the specification,
at the time of writing.}
\end{table}

\section*{Discussion}

The difficulty of describing complex demographic models
for population genetic simulations has long been acknowledged
~\citep[][e.g.]{antao2007modeler4simcoal2}, % jk - any older refs for this?
and a number of different methods have been proposed to
make such simulations more accessible and less error prone.
Several Graphical User Interface (GUI) and visualisation
methods have been developed, which greatly facilitate
interpretation~\citep{mailund2005coasim,antao2007modeler4simcoal2,
ewing2010msms,zhou2018popdemog}. However, these methods
currently have little traction as they are all either directly coupled
to an internal simulation method or to the specific syntax
of a given simulator. Another approach that has been taken to
simplify running such simulations is to provide an API in a
scripting language such as Python or R, providing a flexible
interface either to a built-in simulation
engine~\citep{thornton2014cpp,kelleher2016efficient,haller2017flexible}
% jk: are/were there other simulation frontends out there?
or as a frontend for other simulators~\citep{staab2016coala}.
While APIs have many advantages and have grown in popularity in
recent years, the demographic model descriptions are tied
to the details of both the underlying simulation engines
and the programming language in question.

Stable and healthy software ecosystems require standard interchange
formats, allowing for the development of high-quality and long-lasting
tools that produce and consume the standard.
Demographic models are a key part of population genetics research,
and to date the transfer of inferred models to downstream simulations
has been ad-hoc, and conversions between the many different ways
of expressing such models is both labour intensive and fraught with errors.
The proposed Demes standard is an attempt to bridge this gap
between inference and simulation, and also to provide the foundations
for a sustainable ecosystem of tools built around this data model.
Table~\ref{tab-software} shows some initial infrastructure that we have
built as part of developing Demes, but many other useful tools
can be envisaged that either produce or consume this format.

Reproducibility is a significant problem throughout the
sciences~\citep{baker20161} and various measures have been
proposed to increase the likelihood of researchers being
able to replicate results in the
literature~\citep{munafo2017manifesto}. The most basic requirement
for reproducibility is that we must be able to state precisely what
the result in question is. The lack of standardisation in how
complex demographic models are communicated today, and the lack of
precision in the published model descriptions means that it is difficult
to replicate analyses, or reproduce those models for later simulation.
Thus, we hope that the Demes standard introduced here will be widely adopted
by inference methods and the results reported in publications
routinely included as supplemental material or uploaded to a data
repository in Demes format.

\bibliographystyle{plainnat}
\bibliography{paper}

\end{document}
