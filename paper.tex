\documentclass[11pt]{article}
\usepackage[round,comma,authoryear]{natbib}
\usepackage{fullpage}
\usepackage{authblk}
\usepackage{graphicx}
\usepackage[newfloat]{minted}
\usepackage{tcolorbox}
\usepackage{etoolbox}
\BeforeBeginEnvironment{minted}{\begin{tcolorbox}}
\AfterEndEnvironment{minted}{\end{tcolorbox}}
\usepackage{url}
\usepackage{fancyvrb}
\usepackage{caption}
\newenvironment{code}{\captionsetup{type=listing}\centering}{}
\SetupFloatingEnvironment{listing}{name=Demographic Model}


% local definitions
\newcommand{\msprime}[0]{\texttt{msprime}}
\newcommand{\stdpopsim}[0]{\texttt{stdpopsim}}
\newcommand{\demes}[0]{\texttt{demes}}
\newcommand{\Demes}[0]{\texttt{Demes}}
\newcommand{\YAML}[0]{\texttt{YAML}}
\newcommand{\moments}[0]{\texttt{moments}}
\newcommand{\dadi}[0]{\texttt{$\partial$a$\partial$i}}
\newcommand{\fwdpy}[0]{\texttt{fwdpy11}}
\newcommand{\slim}[0]{\texttt{slim}}

\newcommand{\aprcomment}[1]{{\textcolor{blue}{APR: #1}}}

\begin{document}

\title{\demes: a Python package for pain-free specification of complex demography}
\author[1,*]{Graham Gower}
\author[2,*]{Jerome Kelleher}
\author[3,*]{Aaron P. Ragsdale}
\author[4,*]{Kevin Thornton}
\affil[1]{}
\affil[2]{}
\affil[3]{}
\affil[4]{}
\affil[*]{Authors listed alphabetically}
\maketitle

\abstract{
come back to title and abstract later
}

\section*{Introduction}

The ever-increasing amount of genetic sequencing data from genetically and
geographically diverse species and populations has allowed us to
infer complex demography and study life history at fine scales. An integral
component to such population genetics studies is simulation. Software to
either simulate whole genome sequences
\citep{thornton2014c++,kelleher2016efficient,haller2019slim,adrion2020community}
or informative summary statistics of diversity
\citep{gutenkunst2009inferring,kamm2017efficient,jouganous2017inferring}
have enabled the increasing complexity of genomic studies, with many software
packages able to handle large sample sizes, many interacting populations, and
deviations from random-mating models of panmixia.

The ability to simulate complex demographic scenarios, however, does not come
without its own set of obstacles and frustrations. First of all, specifying models
with many populations and parameters can be cumbersome and error-prone
(e.g., \citep{ragsdale2020lessons}). It is
time-consuming and tedious to implement and then verify the correctness
of such demographic models. Making matters worse, each software package typically has
its own application programming interface (API), which makes it difficult to translate
models between simulation methods.

Secondly, many software API are not particularly easy to read, especially when there
are a large number of demographic events and parameters. This not only makes
debugging difficult, but it can be difficult to even notice if a mistake has been
made. A common interface for unambiguously specifying demographic
models would reduce implementation errors and promote ease-of-use for many simulation
software packages.

Finally, simulated or inferred demographic models are regularly shared in the
literature, and for someone else to be able to reproduce or reimplement some
demographic scenario, it needs to be described fully and without ambiguity. However,
it is common for reported demographic models to lack clear or even complete
descriptions, hindering reproducibility. A human-readable description for
such models would facilitate their sharing and reimplementation.

Here, we present \demes, a Python package for specifying complex demographic models
in a way that is both human readable and can be directly passed to a growing
number of simulation engines. \Demes\ supports model specification in a high-level
\YAML\ format \citep{ben2009yaml}
that is designed to be as easy as possible to implement, read, and
share. Models are then represented in an unambiguous low-level format within Python
that checks model validity and can then be used as the input for simulation software.
The initial release of \demes\ (version 1.0, described here) supports \emph{static}
demographic descriptions, that is, models with fixed parameters. Future releases
will support [drawing parameters from distributions, for example for ABC-based
inference, or describing ``dynamic`` models where parameters may be fit to data,
e.g. \moments\ or \dadi].

\section*{Defining demographic models with \demes}

\subsection*{High-level model specification in \YAML}

\begin{itemize}
\item Short description of YAML language and why it's appropriate here
\item General overview (1 paragraph) of defining demography in YAML
\item A generic, commonly used, simple demographic model for illustration (e.g. IM)
\end{itemize}

\begin{code}
\label{code:im_model}
\captionof{listing}{An IM model.}
\begin{minted}[linenos,numbersep=5pt]{yaml}
description: A two-population isolation-with-migration model.
time_units: generations
demes:
  ancestral:
    description: The ancestral deme that splits into two child demes.
    initial_size: 5000
    end_time: 2000
  deme1:
    description: The first child deme.
    ancestors: ancestral
    initial_size: 1000
    final_size: 10000
  deme2:
    description: The second child deme.
    ancestors: ancestral
    initial_size: 2000
    final_size: 8000
migrations:
  symmetric:
  - demes: deme1, deme2
    rate: 1e-4
\end{minted}
\end{code}

\subsection*{Python representation of demography as input for simulation software}

\begin{itemize}
\item The Python API, reading in a YAML, and unambiguous and manipulable representation
\end{itemize}

\subsubsection*{Software support for \demes}
\aprcomment{at time of publication. TBD...}
\begin{itemize}
\item \msprime
\item \stdpopsim
\item \fwdpy
\item \moments
\item \dadi? should try to get Ryan on board
\item \slim? I guess that comes with \stdpopsim
\end{itemize}

\subsection*{Specifying complex demography}

\aprcomment{A few examples of complex demography - e.g. 1) the very common out-of-Africa
model, 2) Denisovan admixture with a ton of populations and demographic events, 3) a
non-human model, perhaps with selfing rates and rate changes}

\aprcomment{Perhaps examples should be in the appendix, to keep the paper concise}

\section*{Prospects and future directions}

\bibliographystyle{plainnat}
\bibliography{paper}

\section*{Appendix}




\end{document}
